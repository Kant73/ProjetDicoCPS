\section*{Projet C\-P\-S Dictionnaire}

\subparagraph*{Réalisation d'un dictionnaire à partir des mots composant un texte quelconque, écrit en minuscules, sans accent et sans caractères spéciaux.}

\subsection*{Compilation et utilisation}

\subsubsection*{Compilation}

Pour compiler le programme, il suffit de se rendre dans le dossier src, d'ouvrir un terminal et d'executer la commande make qui va généré un executable nommé \char`\"{}main\char`\"{}.

\subsubsection*{Utilisation}

Pout lancer le programme, utilisez le de la façon suivante \-: ``` ./main \mbox{[}nom\-\_\-de\-\_\-fichier\mbox{]} ``` Par exemple \-: ``` ./main ../\-Examples/moyen\-\_\-dico.txt ```

\mbox{[}nom\-\_\-de\-\_\-fichier\mbox{]} correspond au nom du fichier dont l'utilisateur souhaite connaitre le dictionnaire.

Si aucun nom de fichier n'est saisi, l'utilisateur devra alors lui même taper le texte dont il veut connaitre le dictionnaire, c'est le mode \char`\"{}saisie de texte manuelle\char`\"{}.

Exemple \-: ``` quentin-\/\-K72\-Jr\-:$\sim$/\-Documents/polytech/\-C\-P\-S/\-Projet/\-Projet\-\_\-\-Dict/src\$ ./main salut comment ca va ? tres bien et vous ? ca va merci. 

 Dictionnaire \-: 

 bien (1, 28) ca (1, 15) (1, 43) comment (1, 7) et (1, 33) merci (1, 49) salut (1, 1) tres (1, 23) va (1, 18) (1, 46) vous (1, 36) 

 Fin du Dictionnaire, nombre de mots \-: 11 

 ```

\paragraph*{Remarque \-:}


\begin{DoxyEnumerate}
\item Le makefile est actuellement fait pour les systèmes Linux 64 bits, il est nécéssaire de modifier le flag F\-L\-A\-G\-S\-S\-O de celui-\/ci afin de le rendre compatible avec votre système.
\item Si vous optez pour l'utilisation de l'outil en mode \char`\"{}saisie de texte manuelle\char`\"{}, il est nécéssaire de terminer votre saisie par un appui sur la touche Entrée de votre clavier puis par la combinaison Ctrl + D pour les systèmes Linux (un équivalent de Ctrl + D doit exister pour les systèmes Mac O\-S) 
\end{DoxyEnumerate}